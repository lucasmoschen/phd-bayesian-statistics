\vspace{2ex}

\fbox{\parbox{0.92\textwidth}{
    {\bf Solution.} The only restriction prior to the data we have is that $N \in
    \mathbb{Z}^+$. Therefore we are looking for a distribution in this space.
    We can follow the assumption made by \cite{Raftery1988} where $N \sim
    \Poi(\mu)$. Other approach is to consider $N \sim \Geom(\nu)$.

    The first approach has the advantage in calculations. When 
    $$x_i | N, \theta \sim \Bin(N,\theta)$$ and $$N \sim \Poi(\mu),$$ we have that $x_i \sim
    \Poi(\mu\cdot\theta)$, as proved in Appendix \ref{sec:binomial-poisson}. And we can convert prior information in $x_i$ in
    terms of our beliefs about its mean. 

    The second approach serves as a comparative. The interesting part of this
    distribution is the compact domain since $\nu \in [0,1]$. From that, we
    can build a model that caries the correlation between $\nu$ and $\theta$
    whenever necessary in a direct way: we transform the variables to logistic
    space and defines a bivariate normal distribution. We would have to define
    five parameters, but this is a good extension. 

    \vspace{2ex}

    {\bf Priors to the hyperparameters}

    \vspace{2ex}

    Now we shall define the priors to the hyperparameters. Define $\lambda =
    \mu\cdot\theta$. Given that it is morel likely that previous research made
    statements about $x_i$, we use the same idea as Raftery considering the
    prior over $(\lambda, \theta)$. I will assume they are independent from
    now on with 
    $$
    \lambda \sim \DGamma(\alpha, \beta),
    $$
    and
    $$
    \theta \sim \Beta(a,b).
    $$

    The other approach will have two settings. First I suppose $\nu$ and
    $\theta$ independents with 
    $$
    \nu \sim \Beta(\alpha_1, \beta_1)
    $$
    and
    $$
    \theta \sim \Beta(\alpha_2, \beta_2).
    $$
    I choose the Beta distribution because it has a flexible shape with
    a good intuition behind it. Other point is that the Beta distribution
    forms a conjugate family for the Geometric distribution. Another set up is
    to consider the correlated case. We do it in the following way:
    $$
    \begin{pmatrix}
        logit(\nu) \\ logit(\theta) 
    \end{pmatrix} \sim \Normal(\mu, \Sigma).
    $$
}
}

\fbox{\parbox{0.92\textwidth}{
This choice is intrinsically linked to the fact the the normal distribution is
a good approximation to a series of events, and it has a very good
interpretation of the parameters. The problem with this approach is that it is
harder to codify prior information. We necessarily need information about
\begin{enumerate}
    \item[{\bf(i)}] Beliefs about $N$: if we access previous information about
   the mean, we can think about $(1 - \nu)/\nu$.
   \item[{\bf(ii)}] Beliefs about $\theta$. 
   \item[{\bf(iii)}] Beliefs about how these parameters relate. This is the
   harder part, and may have less research.  
\end{enumerate}

From these three approaches, I will call them in the text (1) Raftery
approach, (2) Geometric and independent approach, and (3) Geometric and
correlated approach. 

}}

\vspace{2ex}