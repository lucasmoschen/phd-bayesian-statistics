\vspace{2ex}

    {\bf Solution.} I first observe that both data sets are unstable by the
    above definition. The mean over variance from impala data is 1.6, while
    waterbuck's is 1.3. Both smaller than 1.71. Since the binomial
    distribution has two unknown parameters, I need to calculate two moments to
    obtain the MME. I calculate the sample mean $\hat{\mu}$ and the sample second moment $\hat{\sigma}^2 -
    \hat{\mu}^2$, where $\hat{\sigma}^2$ is the sample variance. Therefore, I build the
    following system of equations 
    \begin{gather*}
        \hat{\mu} = Np, \\
        \hat{\sigma}^2 = Np(1-p),
    \end{gather*} 
    that is 
    $$\hat{\sigma}^2 = \hat{\mu}(1 - p) \implies \hat{p} = 1 -
    \frac{\hat{\sigma}^2}{\hat{\mu}}$$
    and
    $$
    \hat{N} = \frac{\hat{\mu}}{\hat{p}} = \frac{\hat{\mu}^2}{\hat{\mu} - \hat{\sigma}^2}.
    $$
    For instance, the $N_{MME}$ calculated were 56.54 for the impala data set, and 271.84
    for the waterbuck.

    The setup from \cite[Section 3]{Carroll1985} is used to compare the
    estimators, and to analyse the stability. First, I generate eight samples
    with different parameters, and 
    calculate the four Bayes estimates, besides the moment estimator. For each
    sample, I also generated a perturbed sample adding one to the largest
    value.  
    
    
    
    
    
    After that, I chose randomly and uniformly $3 \le K \le 22$, $0 <
    \theta < 1$, and $1 \le N \le 100$. I generate 2000 cases and calculate the
    quadratic loss.   
