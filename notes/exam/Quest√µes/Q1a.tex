\begin{proof}[Solution]
    We want to make inference about $p$, the stopping point of the ball $W$. We know that $0 < p < 1$, so we define the
    parameter space $\Theta = (0,1)$. We are considering $X$ a non-negative
    integer between $0$ and $n$, that is, $\mathcal{X} = \mathbb{Z}_+ \cap
    [0,n]$. 
    
    Each time we roll the ball $O$, there is a uniform probability of
    stopping between $0$ and $L = 1$. Then, 
    $$\Pr(\text{position of } O < \text{position of } W) = \Pr(\text{position of } O < p) = p$$
    and each trial has a probability $p$ of success and $1-p$ of fail. Therefore,
    doing $n$ conditionally independent experiments with probability of
    success $p$, the data-generating mechanism is a Binomial
    distribution with parameters $n$ and $p$, that is,
    $X \sim Bin(n,p)$. 

    The density of the distribution of $X$ with respect to the counting measure is given by the expression 
    $$
    f(x|n,p) = \binom{n}{x} p^x(1 - p)^{n-x}.
    $$
    We'll drop the conditionality of $n$ and suppose it's fixed. Then, the
    likelihood is 
    $$
    L(p|x) =  \genfrac(){0pt}{0}{n}{x} p^x(1 - p)^{n-x}.
    $$

    We defined the essential elements
    for a simple statistical inference. Now we must define the crucial elements of a Bayesian analysis: 

    \begin{enumerate}
        \item[(1)] Prior distribution: $\pi(p) = 1$ (over the space $\Theta$).
        This is because we have prior knowledge that the position of ball $W$ has a uniform probability
        between $0$ and $L = 1$. 
        \item[(2)] Loss function: The problem didn't express it, however  it's
        important to evaluate the quality of our inference. We can consider
        the absolute loss, because we are measuring the distance between
        points in a line. 
    \end{enumerate}

    The dominating measure in $\Theta$ is the Lebesgue measure and the
    dominating measure in $\mathcal{X}$ (which defines the likelihood
    function \cite[]{schervish1996theory}) is the counting measure. 
\end{proof}