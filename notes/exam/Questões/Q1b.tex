\begin{proposition}
    \label{prop:joint_bayes}
    The joint distribution of $p$ and $X$ is given by
    \begin{equation*}
     \pr(a < p < b, X = x) = \int_{a}^{b} \binom{n}{x} t^x(1-t)^{n-x}\,dt.
    \end{equation*}
\end{proposition}

\begin{proof}
    By the definition of conditional density, if $f_{p,X}(p, x)$ is the
    joint density of $p$ and $X$, 
    $$
    f_{p,X}(p, x) = f_{X|p}(x|p)\cdot \pi(p) = f(x|p).$$  
    Then 
    \begin{equation*}
        \begin{split}
            \pr(a < p < b, X = x) &= \pr(p < b, X = x) - \pr(p \ge a, X = x) \\ 
            &= \pr(p < b, X \le x) - \pr(p < b, \le x - 1) \\
            &~~~- \pr(p \ge a, X \le x) + \pr(p \ge a, X \ge x -1) \\
            &= \int_0^b \sum_{i=0}^x f(i|t) \, dt - \int_0^b \sum_{i=0}^{x-1} f(i|t) \, dt \\ 
            &~~~-  
            \int_0^a \sum_{i=0}^x f(i|t) \, dt + \int_0^a \sum_{i=0}^{x-1} f(i|t) \, dt \\
            &= \int_0^b f(x|t)\, dt - \int_0^a f(x|t)\, dt \\
            &= \int_a^b \binom{n}{x} t^x(1 - t)^{n-x} \, dt
        \end{split}
    \end{equation*}
    as we wanted to prove. 
\end{proof}

\begin{proof}[Solution]
    The conditional distribution function $F_{p|X}(b|x)$ is defined by the expression
    $$
    F_{p|X}(b|x) = \frac{\pr(p \le b, X = x)}{\pr(X = x)}, \text{ whenever } \pr(X = x) > 0.
    $$
    By Proposition \ref{prop:joint_bayes}, we can calculate in the following
    manner:
    $$
    F_{p|X}(b|x) = \frac{\int_{0}^{b} \binom{n}{x} t^x(1-t)^{n-x}\,dt}{\int_{0}^{1} \binom{n}{x} t^x(1-t)^{n-x}\,dt} = \frac{\int_{0}^{b} t^x(1-t)^{n-x}\,dt}{\int_{0}^{1} t^x(1-t)^{n-x}\,dt}
    $$
    given that $\pr(X = x) = \pr(p \in [0,1], X = x)$. The posterior measure
    $\mu_{p|X}$ can be defined through $F_{p|X}(b|x)$. The expressions in the
    integrals do not have closed form and the ratio can be expressed as a
    ratio of incomplete beta functions \cite{beta}: 
    $$F_{p|X}(b|x) = \frac{B(b; x+1, n-x+1)}{B(1; x+1, n-x+1)}.$$
    In special, $B(1;x+1,n-x+1) = B(x+1,n-x+1)$ is the beta function. 

    The density of the posterior measure can be calculated using the
    Fundamental Theorem of Calculus \cite{calculus}
    \begin{equation}
        \label{eq:posterior-1b}
        \xi(p|x) = \frac{d}{dp}F_{p|X}(p|x) = \frac{1}{B(x+1,n-x+1)}p^x(1 - p)^{n-x}.
    \end{equation}
\end{proof}